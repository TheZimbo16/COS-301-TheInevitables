
\documentclass{article}
\usepackage[utf8]{inputenc}
\usepackage{graphicx}
\usepackage{geometry}
 \geometry{
	 a4paper,
	 total={170mm,257mm},
	 left=20mm,
 	top=20mm,
 } 
 
\title{3D VR Presentations EPI-USE \\
The Inevitables
}

\author{  
            Peter Rayner\\
            Dawie Pritchard\\
            Drew Langley\\
            Hendrik Jan van der Merwe\\
            Lyle Nel\\
        }


\begin{document}

\maketitle

\includegraphics[width=20cm,height=11cm,keepaspectratio]{group.JPG}

\newpage

\tableofcontents

\newpage


\section{High level description}
This project consists of multiple parts
\subsection{VR Presentation Creator}
Any user should be able to create a VR presentation. This means that it should be designed with user experience and human computer interaction in mind.
an example would be Microsoft's Powerpoint.

\subsection{Presenter Software}
Software that allows a user to easily present a VR presentation. Create at least one 3D VR presentation to work with. The software must send the current scene as well as the presenters voice to multiple observers/listeners and the presenter should be able to switch between them easily. We will also add recording and exporting functionality to create 360 videos that can be uploaded to Youtube.

\subsection{Observer/Listener Software}
Software that allows users to view and experience VR presentations. We will need at least one 360 VR video sent from a server to multiple clients. 


\subsection{System requirements:}
Clients should be able to use various devices such as VR sets, computers, phones or tablets (Cross platform).

This information was taken from EPIUSE3DVRPresentations.pdf

\subsection{Possible Project Name}
Baby Blue

\newpage
\section{Proposed Solution}

\subsection{Technologies}
\begin{itemize}
	\item Unity3D
	\item Oculus rift
	\item MongoDB
\end{itemize}
Note that during the course of the development some technologies may be included/ excluded based on the needs of the system.
\subsection{Deployment Diagram}
\includegraphics[width=20cm,height=11cm,keepaspectratio]{dd2.png} \\

\subsubsection{Description}
We intend to use a microservices architecture which provides loose coupling and high cohesion between each module. Each module will then be split into an N-Tier architecture that can be worked on seperately from the other modules. This project entails 3 modules for each part of the project and each part can be worked on seperately.
\section{Development Methodology}
We will be using the Agile Scrum Methodology. By making a backlog of work to be done and by completing deadlines in short iterations or sprints. We will meet daily by making use of  Slack Messaging Platform to discuss the progress as well as obstacles and how to overcome these obstacles by getting input from each member. We will also define when these deadlines are and make sure we keep to the schedule. We will meet everytime we are done with a deadline to reflect on the work done. \\ \\
At each deadline or meeting we will make sure we meet with the client to make sure that they are kept up to date with our progress. We will also meet with the client when there are concerns or obstacles to overcome to make sure the client knows  about these obstacles. The client will be kept up to date each week with the progress of the project.
\section{Cost Analysis}
(To be calculated)


\section{Risk Analysis}
\subsection{Experience}
None of us have worked with VR before. However three students are from the multimedia department and from those 3 students. One has completed IMY 300 and two are currently busy with it,the course entails game development, HCI aspects, UX aspects, 3D modelling as well as animation. Two other members have no experience  with 3D modelling, UX, HCI and animation and thus will require them to perform some research.


\subsection{Security}
The system has to be designed in such a way that observers/listeners will not be interrupted by other observers/listeners that should not have access, thus the system has to have login/logout capabilities and information has to be protected from the observers/listeners such as passwords.

\newpage
\section{Team Details}
\subsection{Dawie Pritchard}
\textbf{Skills:}
\begin{itemize}
 	\item Human Computer Interaction
 	\item Trends, Visual Design 
 	\item Multimedia Specialist
 	\item Computer Scientist
\end {itemize}
\textbf{Technologies known:}
\begin{itemize}
	\item C
 	\item C++
 	\item C-Sharp
 	\item CSS
 	\item Bootstrap
 	\item Java
 	\item Python
 	\item Javascript / AngularJS / ExpressJS / NodeJS / JQUERY
 	\item MongoDB / NoSQL
 	\item Php
 	\item SQL
 	\item HTML5
 	\item XML
 \end{itemize}
\textbf{Stengths:} 
\begin{itemize}
	\item Front-End
	\item Back-End development
\end{itemize}

\newpage
\subsection {Peter Rayner}

\textbf{Front-end developer with knowledge of:}
\begin{itemize}
 	\item Artifical Intelligence
 	\item Data structures 
 	\item Website design 
 	\item Databases and human computer interaction(user experience)
 \end{itemize}
\textbf{Technologies known:}
\begin{itemize}
	\item C++ 
	\item C 
	\item C\# 
	\item Java 
	\item JavaScript 
	\item Python 
	\item Assembly 
	\item AngularJS  
	\item Bootstrap
 \end{itemize}
\textbf{Previous industry experience:}\\
Working at Barclays CIB in big data and analytics.
\\
\newpage
\subsection {Hendrik Jan van der Merwe} 
\textbf{University level knowledge of:}
\begin{itemize}
 	\item Data structures
 	\item Databases
 	\item Human Computer Interaction focussing on User Experience
 	\item System Design
\end{itemize}
\textbf{Technologies known:}
\begin{itemize}
	\item C++
	\item C\#
	\item Java
	\item SQL / MySQL
	\item MongoDB
	\item PHP
	\item JavaScript / AJAX / JQuery / NodeJS / ExpressJS
	\item HTML / CSS / Bootstrap
	\item XML
\end{itemize}
\textbf{Strengths:}
\begin{itemize}
	\item Database Design
	\item Backend Development
\end{itemize}

\newpage
\subsection {Lyle Nel}
\textbf{Qualifications:} \\
I hold a BTEC in software engineering, which included project management as part of the curriculum.
I also hold a BSc in Computer Systems, with relevant subjects such as Software Engineering, Operations management, Knowledge management, Professional development, and Artificial intelligence.\\ \\
\textbf{Digital electronics:} \\
I have worked with Atmel and ARM microprocessors as well as on the arduino platform. In addition I am familiar with most of the common components of a digital circuit including 7400 series and 4000 series integrated circuits. \\ \\
\textbf{Computer Hardware:} \\
I am familiar with all standard consumer hardware and some server hardware. I maintain my own server cluster at home for running experiment. \\ \\
\textbf{Artificial intelligence} \\
I am most experienced in genetic algorithms and I am the author of an open source project that cracks passwords using genetic algorithms. The was one of the top 3 trending projects on github and hackernews a while back. See https://github.com/lyle-nel/siga. In general I am very comfortable with solving problems within the domain of AI. \\ \\
\textbf{Languages} \\
C, C++ including the new C++11, C++14 and C++17 ISO standards, Bash, Python, Javascript, Java, Lisp and Prolog. The language that I am most comfortable with is C++. When I conduct experiment on large datasets I use a mixture between C++, Python and Bash. \\ \\
\textbf{Platforms} \\
I do all of my work in a Linux environment.

\newpage
\subsection {Drew Langley}
Third year BIS Multimedia student with experience in UX and HCI, animation and 3D modelling, Web design and databases as well as proficiency in programming. I am currently studying networks, software engineering and Artificial Intelligence. \\ \\
\textbf{Technologies known:}
\begin{itemize}
	\item HTML
	\item CSS 
	\item JS / AngularJS / NodeJS / JQuery 
	\item PHP 
	\item SQL
	\item MongoDB
	\item Java 
	\item C++ 
	\item C 
	\item C\# 
	\item Python 
	\item Assembly
\end{itemize}
\textbf{Experience:} \\
Designed and developed www.ugandaprohunts.com

\textbf{Stengths:} 
\begin{itemize}
	\item Front-End
	\item Back-End development
\end{itemize}

\end{document}