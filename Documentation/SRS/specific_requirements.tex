%%%%%%%%%%%%%%%%%%%%%%%%%============= Specific Requirements ================%%%%%%%%%%%%%%%%%%%%%%%%%
\section{Specific Requirements}

	\subsection{External Interface Requirements}
		\subsubsection{User Interfaces}
		        The mobile application will interface with the supported input and output features of the host's operating system. Inputs include text that the user will enter for login or searching a venue. Outputs include the type of fonts to display text or graphics to show images or draw the map.

		\subsubsection{Hardware Interfaces}
			Since neither the mobile application nor the web portal have any designated hardware, it does not have any direct hardware interfaces. The WiFi software in the mobile phone manages the built-in WiFi and the hardware connection to the database server is managed by the underlying operating system on the mobile phone and the web server.

		\subsubsection{Software Interfaces}
		The final prototype will likely make use of the OpenCV API for machine vision, while consulting a standard PostgreSQL database system for persistent data such as GIS information. Some of these queries are triggered by interaction with the front-end user interface such as viewing the map and finding the shortest route, while others are automated such as fiducial marker identification.

		\subsubsection{Communication Interfaces}
			The communication between the different parts of the system are important since they depend on each other. However, in what way the communication is achieved is not important for the system and is therefore handled by the underlying operating systems for both the mobile application and the back-end of the system.
	
		
	\subsection{Performance Requirements}
		\subsubsection{Performance}
			\begin{itemize}
			\item Offline activities should have a response time of +/- 2 seconds (instantaneous) when responding to an activity, while online activities such as calculating routes should have a response time of +/- 2-4 seconds so that the users have an uninterrupted experience.
			\item It should also allow the integration of a variety of services.\\
			\end{itemize}
			\subsubsection{Reliability} 
			\begin{itemize}
			\item It should be able to handle +/- 50 000 users concurrently (simultaneously) when implemented into a suitable production environment. 
			\item The application should be reliable, in that it will provide the fastest route every time without fail and complete all other computations successfully. 
			\item All activities should be completed with a 0.1\% error allowance.
			\item The application should provide accurate locations in a constantly changing environment.\\
			\end{itemize}
			\subsubsection{Security}
			\begin{itemize}
			\item Data transmission should be securely transmitted without unauthorized access, or loss of information.\\
			\end{itemize}

	
	
	\subsection{Design Constraints}
		\begin{itemize}
				\item Security  - The users personal information and current location should not be accessible to the public.
				\item Accuracy - The users location should be found whether the user is indoors or not. The location of the user should also be found in terms of which floor they are on in the building.
				\item Performance - The system should be able to handle a large amount of users making use of the software at the same time, the system should also use resources on the client device efficiently.
				\item Reliability - The application should still operate when one or more data access points are no longer available.
				\item Accessibility - The application should be easily accessible to everyone that requires it.
				\item Usability - The applications interface should be easy to use and understand.
				\item Size - The application should not require too much memory in order to operate.
				\item Users - The nature of some users enforces constraints such as conforming to the WCAG for accessibility issues.
				\item Sensors - A built in digital camera, GPS and magnetometer is needed, all of which is standard in most smart phones.
				\item Platform - the system should be accessible on  Android  devices.
				\item Modularity - The system should be modular, thus allowing for high cohesion with low coupling and reducing the dependencies within the system.
				\item Aesthetics - The system's interfaces should be aesthetically pleasing and follow the UP branding guidelines.
				\item Fault tolerance - if unavoidable malfunctions occur,the design should be constrained such that the system can recover without loss of data or damage.
				
			\end{itemize}
	
	\subsection{Software System Attributes}

		\subsubsection{Reliability}

			\begin{itemize}
				\item Any information that is stored on the database must remain correct when being transferred to the user interface.
				\item The services offered by the system should be available to users except for when the system is undergoing maintenance.
				\item The system should reply to user requests in the shortest time interval possible.
				\item The system must be fault tolerant, it needs to maintain a certain level of performance and offer other services that are not affected by this fault to the users.
				\item In the event of a fault the system must be able to recover within the shortest time period possible and recover any data that may have been lost.
				\item The system should be able to respond appropriately if it receives bad input data from the user.
			\end{itemize}

		\subsubsection{Scalability}
			\begin{itemize}
				\item The system must be able to cater for increases in the work load, for example large number of users or activities at any given time, without impacting the performance of the system.
				\item If the system does not cater for increases in workload it should at least provide the ability to be readily enlarged.
			\end{itemize}
		
		\subsubsection{Maintainability}
			\begin{itemize}
				\item The system must be designed in a modular fashion that provides high cohesion and loose coupling, this will allow parts of the system to be easily maintained without affecting the rest of the system.
				\item Maintenance should be able to be carried out by different maintenance teams, therefore the system must be easy to learn and understand.
			\end{itemize}

		\subsubsection{Integrability}
		
			\begin{itemize}
				\item Since we are following a modular design, components of the system that are separately developed should work correctly together.
				\item Follow coding standards specified by the client to allow for easy integration and employ continuous integration in our design process.
			\end{itemize}
		
		\subsubsection{Usability}
		
			\begin{itemize}
				\item The system must be easy to learn.
				\item System must cater for user mistakes, by providing the user with the undo or roll back options.
				\item The user interface must be easy to use and must be intuitive.
				\item The system should display options in a logical manner.
				\item Incorporate widgets and icons that the target users may be familiar with.
				\item The user manual should have a detailed description of the system.
				\item A help option must be provided to the users.
			\end{itemize}

