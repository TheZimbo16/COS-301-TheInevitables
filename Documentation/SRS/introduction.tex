%%%%%%%%%%%%%%%%%%%%%%% =========== INTRODUCTION ================== %%%%%%%%%%%%%%%%%%%%%%%%%%%%

\section{Introduction}
		Many mobile applications exist to assist their users in successful travel from one location to another, by vehicle, or on foot. \\
		Though these applications exist and work very well, solutions for disabled persons such as visual and physical impairments do not exist.\\ \par
		This document serves to formally introduce and quantify the Object Sensor and Mobile Navigation application, this project is part of the COS 301 (Software Engineering) Capstone project available for bid in 2017. This project was successfully tendered by TheInevitables.
	
	\subsection{Purpose}
		The purpose of this document is to provide a detailed description of the requirements presented by the Object Sensor and Mobile Navigation application, this application is intended to assist the students of the disability unit with navigation around the University of Pretoria. Additionally this document will include possible constraints and technical requirements such as how the system wil interact with external applications, such as the users. \par
		This document is intended to comply with the IEEE 830-1998 SRS Standard. This document is also intended for use by clients Morkel Theunissen and Maria Ramaahlo for approval such that the system implementation may begin.
		
	\subsection{Scope}
%		The orientation and navigation of students with visual disabilities is essential to them accessing the academic environment. Whilst, many students with visual disabilities may make use of trained guide dogs or canes, additional support is required when navigating their way through the many obstacles on campus. This includes but is not limited to poles, low hanging roofs,bollards and other objects.\\
%		The Object Sensor and Mobile Navigation application will make use of GPS, GIS and WiFi access point data to determine the current location of a user and safely navigate them to their destination on the University of Pretoria's campus facilities, this includes but is not limited to lecture halls, laboratories, food courts, ablutions and so on. The system will also make use of fiducial markers which will be used to notify the user of an impending obstacle.

		
	\subsection{Definitions, Acronyms and Abbreviations}
		\begin{itemize}
			\item IEEE - Institute of Electronic and Electrical Engineers.
			\item GPS - Global Positioning System.
			\item GIS - Geographic Information System.
			\item WiFi - Wireless Fidelity.
			\item WCAG - Web Accessibility Guidelines.
			\item AR Tag - An Augmented Reality Tag that is printable and facilitates physical tagging to be read using machine vision at a later stage.
			\item ArUco - A specific binary square fiducial marker implementation that that is the main candidate for AR Tagging in this project.
			\item Fiducial marker - A machine readable object in the physical world that is used to tag objects that are deemed relevant the navigation system.
		\end{itemize}
		
		From here on the DU@UP Object Sensor and Mobile Navigation application will be referred to as NavUP.
		
	\subsection{References}
		\begin{itemize}
			\item IEEE 830-1998 SRS Standard - \url{https://standards.ieee.org/findstds/standard/830-1998.html}
			\item DU@UP Object Sensor and Mobile Navigation - document supplied by clients, before project tender.
		\end{itemize}
	
\newpage
%	\subsection{Project background}
%		The orientation and navigation of students with visual disabilities is essential to them accessing the academic environment. Whilst, many students with visual disabilities may make use of trained guide dogs or  canes, additional support is required when navigating their way through the many obstacles on campus. This includes but is not limited to poles, low hanging roofs ,bollards and other objects.
%
%	\subsection{Project vision}
%The development of mobile communications allows for the development of applications for the independent navigation of persons with visual disabilities. The combination of mobile technologies, navigation systems and low tech devices will assist students to safely and successfully navigate their way on campus.	
%	
	\subsection{Overview}
		These are the sections that follow; \\
		The Overall Description, which describes the overall system, the modules and the interfaces that exist within the system, the system functions, the characteristics of the users, the systems' possible constraints and the assumptions and dependencies required for the system.\par\noindent
		The Functional Requirements section lists the requirements the system should adhere to and describes use cases.\par\noindent
		The Specific Requirements section is mainly intended for developers and is dedicated to defining the external interface requirements, functional requirements, performance requirements, design constraints, quality requirements and software system attributes. \par\noindent
		The development of mobile communications allows for the development of applications for the independent navigation of persons with visual disabilities. The combination of mobile technologies, navigation systems and low tech devices will assist students to safely and successfully navigate their way on campus.
