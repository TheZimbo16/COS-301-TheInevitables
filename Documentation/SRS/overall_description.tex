%%%%%%%%%%%%%%%%%%%%%%============== OVERALL DESCRIPTION ==============%%%%%%%%%%%%%%%%%%%%%%%%%%%%
\section{Overall Description}

	\subsection{Product Perspective}
		Similar to existing navigation applications such as Google Maps and Waze, the NavUP is a standalone mobile application that acts as a campus navigation system for students both with and without visual disabilities. It encompasses many technologies which will be required to provide the requested and necessary functionality as described in the functional requirements section.\\ \par\noindent
		Obstacles described previously such as poles and low hanging roofs, intersections on campus(between IT Building and EB building) will be fitted with fiducial markers. These markers will be detected using machine vision and will be used to warn users when they are in proximity to certain obstacles or landmarks.\par\noindent
		The system will use the user's smart device's hardware interfaces such as GPS to accurately get the user's location data, and navigate them to their required destination.\\
		
	\subsection{Product Functions}
		The NavUp system will support a variety of functions, the main function focuses on reliable, accurate navigation and presentation of geographical data, both indoors and outdoors. Other functionality includes providing information, and special routes for students with disabilities.
		\\ \par\noindent
		The navigational functionality will consist of getting the user's current destination accurately and then determining a suitable route to the venue the user requests.\\ \par\noindent
		Students with visual disabilities already have defined routes which the system must take into account.\\ \par\noindent
		Information provided to users will consist of venue information, geographical information, and information regarding the proximity of obstacles to visually impaired users.\\ \par\noindent
		Due to the project focusing on the assistance of visually  impaired students and users, the application will conform to the WCAG to ensure these users with ease of use and accessibility.\\ \par\noindent
		The system will be developed to run on mobile devices, and will be acquired by users from the "Google Play Store".\\ \par\noindent
		The application will be perceivable, operable, reliable and follow the University of Pretoria's colour scheme and branding guidelines visually.\\ \par\noindent
		Users without disabilities will have the option to notify the system of a potential hazard or obstacle to visually impaired students. This will be accomplished by allowing them to drop a location pin and will work as a crowdsourcing technique to identify and capture dynamic obstacles. \\ \par\noindent
		The application will also be secure to avoid theft or loss of data.
			
	\newpage
	\subsection{User Characteristics}
		The main category of individuals that will make use of NavUP will be designed for will be visually impaired students of the University of Pretoria.\\
		The main body of students and staff will also have access to the application.
		These students will interact with the system via the mobile application. This means they will not require information regarding how the software works, however it is important to keep their information safe. They will also be required to have access to and knowledge of how to use a smart device.\\ \\
		Other users would include maintenance workers and administrative users, responsible for updating information pertaining to the application. These administrators will require knowledge about how the software works so as to maintain it correctly.
	
	
	
	\subsection{Constraints}
		The following is a list of possible design constraints related to NavUP:
		
			\begin{itemize}
				\item Security  - The users personal information and current location should not be accessible to the public.
				\item Accuracy - The users location should be found whether the user is indoors or not. The location of the user should also be found in terms of which floor they are on in the building.
				\item Performance - The system should be able to handle approximetly 30 000 students making use of the application over the course of an average day.
				\item Reliability - The application should still operate when one or more data access points are no longer available.
				\item Accessibility - The application should be easily accessible to everyone that requires it.
				\item Usability - The applications interface should be easy to use and understand.
				\item Size - The application should as resource frugal as possible.
				\item Users - The nature of some users enforces constraints such as conforming to the WCAG for accessibility issues.
			\end{itemize}
	
	
	\subsection{Assumptions and Dependencies}
		The application will be developed to operate from a user's smart device, these smart devices have all the necessary hardware fitted. It is therefore assumed that all users of the NavUP will have access to one of these smart devices. It is also assumed that these devices have the required hardware dependencies and capabilities to run the application correctly. Such dependencies and hardware include WiFi, GPS and cellular network data connectivity.\\ \\
		It is also assumed that the users will be students of the University of Pretoria, thereby having access to the information regarding their venues.\\
		In terms of usability, it is assumed the users will have knowledge regarding how to use mobile devices and their applications, this includes activating voice controls and so on. \\ \\
		Assumptions relating to the system itself include the assumption that the system will be kept up to date with the latest venue information and geoinformational data. It is also assumed that the fiducial markers will always be operational. These dependencies relate to the accuracy and reliability of the system and it is therefore assumed and recommended that a maintenance schedule be put in place.\\
		A final assumption relating to continuous, reliable performance of the application is dependent on server capacity.

