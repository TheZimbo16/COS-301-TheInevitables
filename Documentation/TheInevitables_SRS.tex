%
% Latex Document made by TheInevitables for SRS,  COS 301 2017
%


\documentclass{article}

\usepackage{geometry}
\usepackage[utf8]{inputenc}
\usepackage{graphicx}
\usepackage{float}
\usepackage{amsmath}
\usepackage{amsfonts}
\usepackage{amssymb}
\usepackage{graphicx}
\usepackage{float}
\usepackage[explicit]{titlesec}
\usepackage{ulem}


\DeclareGraphicsExtensions{.png}
\DeclareGraphicsExtensions{.jpg}
\graphicspath{ {Diagrams/} }

 \geometry{
 a4paper, 
 total={170mm, 257mm}, 
 left=25mm, 
right=25mm, 
 top=25mm, 
 }
 
 \title{ Software Requirements Specification \\ COS-301 \\ The Inevitables \\[0.5cm] \includegraphics[width=6cm]{front-page}}
 
 \author{Drew Langley \hfill 11039753 \\ Lyle Nel \hfill 29562695 \\ Dawie Pritchard \hfill 13104340 \\  Peter Rayner \hfill [Student Number]\\ Hendrik Jan van der Merwe \hfill 15101283 }
\date{23 May 2017}

\begin{document}
\maketitle
\pagebreak
\tableofcontents
\pagebreak


\section{Introduction}
\subsection{Project background}
The orientation and navigation of students with visual disabilities is essential to them accessing the academic environment. Whilst, many students with visual disabilities may make use of trained guide dogs or white canes, additional support is required when navigating their way through the many obstacles on campus. This includes but is not limited to poles, low hanging roofs and other objects.

\subsection{Project vision}
The development of mobile communications allows for the development of applications for the independent navigation of persons with visual disabilities. The combination of mobile technologies, navigation systems and low tech devices will assist students to safely and successfully navigate their way on campus.

\section{System Description}
\section{Functonal Requirements}
\begin{itemize}
	\item Navigation through the use of WiFi Access point data
	\item Functionality across a broad range of mobile devices and operating systems
	\item The application should be developed to function on Apple iTunes and Google Play stores
	\item Compliance with most recent web accessibility guidelines
	\item The application should be perceivable, operable, understandable and robust
	\item Obstacle sensors should be Bluetooth enabled
	\item Has to make use of voice recognition technology
	\item Obstacle sensors should determine localization and distance from user device
	\item Obstacle sensors should notify users of impending collision
\end{itemize}

\subsection{Use Cases}

\subsection{Entity Relationship Diagrams}

\section{External Interface Requirements}
\section{Technical Requirements}
\subsection{Performance}
\subsection{Scalability}
\subsection{Security}
\subsection{Maintainability}
\subsection{Usability}
\subsection{Availability}

\end{document}